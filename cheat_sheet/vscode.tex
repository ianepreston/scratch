%%%%%%%%%%%%%%%%%%%%%%%%%%%%%%%%%%%%%%%%%
% Cheatsheet
% LaTeX Template
% Version 1.0 (12/12/15)
%
% This template has been downloaded from:
% http://www.LaTeXTemplates.com
%
% Original author:
% Michael Müller (https://github.com/cmichi/latex-template-collection) with
% extensive modifications by Vel (vel@LaTeXTemplates.com)
%
% License:
% The MIT License (see included LICENSE file)
%
%%%%%%%%%%%%%%%%%%%%%%%%%%%%%%%%%%%%%%%%%

%----------------------------------------------------------------------------------------
%	PACKAGES AND OTHER DOCUMENT CONFIGURATIONS
%----------------------------------------------------------------------------------------

\documentclass[11pt]{scrartcl} % 11pt font size

\usepackage[utf8]{inputenc} % Required for inputting international characters
\usepackage[T1]{fontenc} % Output font encoding for international characters

\usepackage[margin=0pt, landscape]{geometry} % Page margins and orientation

\usepackage{graphicx} % Required for including images

\usepackage{color} % Required for color customization
\definecolor{mygray}{gray}{.75} % Custom color

\usepackage{url} % Required for the \url command to easily display URLs

\usepackage[ % This block contains information used to annotate the PDF
colorlinks=false, 
pdftitle={Cheatsheet}, 
pdfauthor={Ian Preston}, 
pdfsubject={Compilation of useful shortcuts}, 
pdfkeywords={Random Software, Cheatsheet}
]{hyperref}

\setlength{\unitlength}{1mm} % Set the length that numerical units are measured in
\setlength{\parindent}{0pt} % Stop paragraph indentation

\renewcommand{\dots}{\ \dotfill{}\ } % Fills in the right amount of dots

\newcommand{\command}[2]{#1~\dotfill{}~#2\\} % Custom command for adding a shorcut

\newcommand{\sectiontitle}[1]{\paragraph{#1} \ \\} % Custom command for subsection titles

%----------------------------------------------------------------------------------------

\begin{document}

\begin{picture}(297,210) % Create a container for the page content

%----------------------------------------------------------------------------------------
%	TITLE SECTION 
%----------------------------------------------------------------------------------------
\put(10,200){ % Divide the page
\begin{minipage}[t]{85mm} % Create a box to house text
\sectiontitle{VS Code}
\texttt{\\General\\}			
\command{Ctrl + shift + P, F1}{Show Command Palette}
\command{Ctrl + P}{Quick Open, Go to File...}
\command{Ctrl + shift + \\}{Jump to matching bracket}
\command{Ctrl + shift + [}{Fold region}
\command{Ctrl + shift + ]}{Unfold region}
\command{Ctrl + shift + M}{Show Problems Panel}
\command{F8}{Go to next error or warning}
\command{F3 / Shift + F3}{Find Next/Previous}
\command{Ctrl + Alt $\uparrow \downarrow$}{Insert cursor above/below}
\command{Ctrl + U}{Undo cursor operation}
\command{Ctrl + Space}{Trigger suggestion}
\command{Ctrl + Shift + Space}{Trigger Parameter Hints}
\command{Shift + Alt + F}{Format Document}
\command{F12}{Go to definition}
\command{Alt + F12}{Peek definition}
\command{Ctrl + K F12}{Open Definition to the side}
\command{Shift + F12}{Show References}
\command{F2}{Rename symbol}
\command{Ctrl + K P}{Copy path of active file}
\end{minipage}}

%----------------------------------------------------------------------------------------
%	SECOND COLUMN SPECIFICATION 
%----------------------------------------------------------------------------------------

\put(105,200){ % Divide the page
\begin{minipage}[t]{85mm} % Create a box to house text
\texttt{\\Editor View\\}			
\command{Ctrl + $\backslash$}{Split editor}
\command{Ctrl + 1 / 2 /3}{Focus on nth editor group}
\command{Ctrl + shift + PgUp / PgDn}{Move editor left/right}
\command{F11}{Toggle full screen}
\command{Ctrl + B}{Toggle sidebar visbility}
\command{Ctrl + K Z}{Toggle zen mode}
\end{minipage}}

%----------------------------------------------------------------------------------------
%	THIRD COLUMN SPECIFICATION 
%----------------------------------------------------------------------------------------

\put(200,200){ % Divide the page
\begin{minipage}[t]{85mm} % Create a box to house tex
\texttt{\\Debugging\\}			
\command{F9}{Toggle breakpoint}
\command{F5}{Start/Continue}
\command{Shift + F5}{Stop}
\command{F11 / Shift + F11}{Step into/out}
\command{F10}{Step over}
\end{minipage}}
\end{picture}
\end{document}

